\cvsection{Skills}
\begin{itemize}
\item \textbf{Language}: Python, Clojure, JavaScript, Bash, R, HTML, CSS, French (some working proficiency)
\item \textbf{Technologies}: Dash, Plotly, Spark, Databricks, MapBox, pandas, scikit-learn
\end{itemize}

\cvsection{Education}
\cvevent{McGill University}{Bachelor of Arts}{September 2016--April 2020}{}
\begin{itemize}
\item Double major in Computer Science and English, cultural studies
\end{itemize}

\cvsection{Leadership}
\cvevent{President \& Team Development}{\href{http://www.cicsailing.ca/}{Canadian Intercollegiate Sailing} -  governing body for Canadian collegiate-level sailing}{January 2017 -- December 2019}{}
\begin{itemize}
\item Elected by a coalition of 18 teams across Canada to two terms as president of CICSA and one term as team development officer
\item Oversaw development of an online scoring system built entirely from scratch and implemented a program to provide bursaries for underfunded teams in the league
\end{itemize}

\cvsection{Personal Projects}
\cvproject{Network Analysis of Horror Films}{https://github.com/ndrezn/horror-social-networks}{\githubsymbol}
\begin{itemize}
\item  Studied themematic patterns, tropes, and the structure of the horror genre using social networks
\item Project overseen by Prof. Ned Schantz
\end{itemize}
\cvproject{Wikipedia Picture of the Day}{}{}
\begin{itemize}
\item Daily automated posting of Wikipedia's picture of the day on Instagram using both the Wikipedia API and Instagram API on \href{http://instagram.com/wikipictureoftheday}{@wikipictureoftheday}
\end{itemize}


\cvsection{Publications}
\cvproject{Reconciliation in Chaos}{https://github.com/ndrezn/Palace-of-the-End}{\githubsymbol}
\begin{itemize}
\item  Published in \textit{The Channel}, McGill Department of English Undergraduate Review "\textbf{Reconciliation in Chaos: Tracing word distributions and sentiment in the monologues of \textit{Palace of the End}}", \textit{The Channel}, March 26 2019 \href{http://mcgillchannelundergraduatereview.com/2019/03/reconciliation-in-chaos-tracing-word-distributions-and-sentiment-in-the-monologues-of-palace-of-the-end/}{www.mcgillchannelundergraduatereview.com}
\end{itemize}
