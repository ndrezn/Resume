\cvsection{Skills}
\begin{itemize}
    \item \textbf{Language}: Python, Javascript, Bash, Clojure, French (some working proficiency)
    \item \textbf{Technologies}: Dash, Plotly, CI/CD pipelining, React, Redis, ArcGIS, MapBox, pandas
\end{itemize}

\cvsection{Education}
\cvevent{McGill University}{Bachelor of Arts}{Sept. 2016--Apr. 2020}{}
\begin{itemize}
    \item Double major in Computer Science and English, Cultural Studies.
    \item Awarded the Arts Undergraduate Research Award for .txtLab work, a grant given
          to support undergraduate students who undertake research under the direct
          supervision of a faculty member.
\end{itemize}

\cvsection{Personal Projects}
\cvproject{Wikipedia Histories}{https://github.com/ndrezn/wikipedia-histories/}{\githubsymbol}
\begin{itemize}
    \item A Python tool for retrieving the textual revision history of any Wikipedia
          article, with more than 30,000 lifetime downloads and 200 downloads per week.
\end{itemize}
\cvproject{Wikipedia Picture of the Day}{https://github.com/ndrezn/wikipedia-pic-of-the-day/}{\githubsymbol}
\begin{itemize}
    \item Daily automated posting of Wikipedia's picture of the day on Bluesky at
          \href{https://bsky.app/profile/wiki-potd.bsky.social}{@wiki\_potd} with 50k+
          followers.
\end{itemize}

\cvsection{Publications}
\cvproject{Predicting the Author}{}{}
\begin{itemize}
    \item  Published in \textit{The Channel}, McGill Department of English Undergraduate
          Review "\textbf{Predicting the Author: A Critique of Auteur Theory}",
          \textit{The Channel}, January 18, 2022.
          \href{http://mcgillchannelundergraduatereview.com/2022/01/predicting-the-author-a-critique-of-auteur-theory//}{mcgillchannelundergraduatereview.com}
\end{itemize}
\cvproject{Everyday Specialization on Wikipedia}{https://github.com/ndrezn/wikipedia-histories/tree/network-analysis}{\githubsymbol}
\begin{itemize}
    \item Published by the McGill .txtLab. \textbf{"Everyday Specialization: The
              coherence of editorial communities on Wikipedia"}, \textit{.txtLab
              Collaborations}, September 30, 2020.
          \href{https://txtlab.org/2020/09/do-wikipedia-editors-specialize/}{txtlab.org}
\end{itemize}
\cvproject{Reconciliation in Chaos}{https://github.com/ndrezn/Palace-of-the-End}{\githubsymbol}
\begin{itemize}
    \item  Published in \textit{The Channel}, McGill Department of English Undergraduate
          Review "\textbf{Reconciliation in Chaos: Tracing word distributions and
              sentiment in the monologues of \textit{Palace of the End}}", \textit{The
              Channel}, March 26, 2019.
          \href{http://mcgillchannelundergraduatereview.com/2019/03/reconciliation-in-chaos-tracing-word-distributions-and-sentiment-in-the-monologues-of-palace-of-the-end/}{mcgillchannelundergraduatereview.com}
\end{itemize}
