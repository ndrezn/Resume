% TO COMPILE THIS RESUME: latexmk -pdf -pvc resume.tex 
\documentclass[10pt,ragged2e]{altacv}

% Change the page layout if you need to
\geometry{left=2cm,right=10cm,marginparwidth=6.8cm,marginparsep=1.2cm,top=1.25cm,bottom=1.25cm}

\usepackage[utf8]{inputenc}
\usepackage[T1]{fontenc}
\usepackage[default]{lato}
\usepackage{biblatex}
\usepackage{hyperref}
\hypersetup{
    colorlinks=true,
    linkcolor=blue,
    filecolor=magenta,      
    urlcolor=cyan,
}
 
\urlstyle{same}

% Change the colours if you want to
\definecolor{VividPurple}{HTML}{000000}
\definecolor{SlateGrey}{HTML}{2E2E2E}
\definecolor{LightGrey}{HTML}{2E2E2E}
\colorlet{heading}{VividPurple}
\colorlet{accent}{VividPurple}
\colorlet{emphasis}{SlateGrey}
\colorlet{body}{LightGrey}

% Change the bullets for itemize and rating marker
% for \cvskill if you want to
\renewcommand{\itemmarker}{{\small\textbullet}}
\renewcommand{\ratingmarker}{\faCircle}

\begin{document}
\name{Nathan Drezner}
\tagline{}
\personalinfo{
  \email{\href{mailto:nathan.drezner@mail.mcgill.ca}{nathan.drezner@mail.mcgill.ca}}
  \email{\href{mailto:nathan@drezner.com}{nathan@drezner.com}}
  \github{\href{https://github.com/ndrezn}{ndrezn}}
  \linkedin{\href{https://www.linkedin.com/in/nathan-drezner/}{Nathan Drezner}}
}

%% Make the header extend all the way to the right, if you want.
\begin{fullwidth}
\makecvheader
\end{fullwidth}

%% Depending on your tastes, you may want to make fonts of itemize environments slightly smaller
\AtBeginEnvironment{itemize}{\small}

\cvsection[auxilleryInfo]{Experience}

\cvevent{Research Assistant}{\href{https://txtlab.org}{.txtLab} - a laboratory for cultural analytics at McGill University}{October 2018 -- Present}{Montreal, Canada}
\begin{itemize}
\item Designed and oversaw a .txtLab project to study the salient language of disagreement and consensus building on Wikipedia
\item Conducted data mining of Wikipedia using its API and \textbf{BeautifulSoup} and web scraping of Amazon using BeautifulSoup
\item Used \textbf{sklearn}, \textbf{pandas}, and \textbf{igraph} in \textbf{Python} and \textbf{R} to conduct network analysis and use NLP and ML to study linguistic patterns
\item Project overseen by Profs. Andrew Piper and Richard Jean So
\end{itemize}

\divider

\cvevent{Research Assistant}{\href{https://news.library.mcgill.ca/tag/timeless-riddles/}{Timeless Riddles} - a ROAAr project dedicated to transcribing, solving, and researching riddles found in early manuscripts}{April 2019 -- Present}{Montreal, Canada}
\begin{itemize}
\item Oversaw digital humanities analysis of riddles using in \textbf{Python} by classifying riddles and visualizing semantic change based on riddle periods, publishers, and locations
\item Designed and built a web app to illustrate riddle movement over time with \textbf{MapBox} and \textbf{JavaScript}
\item Worked with tools and datasets from \textbf{Gale} and \textbf{HathiTrust}
\item Project overseen by Prof. Nathalie Cooke
\end{itemize}

\divider

\cvevent{Research Assistant}{\href{https://mcqll.org/}{McGill Computational \& Quantitative Linguistics Lab} - a labratory to study fundamental questions about language acquisition, processing, use in society, and change over time}{November 2019 -- Present}{Montreal, Canada}
\begin{itemize}
\item Contributor for a project studying patterns of productivity and reuse in language
\item Currently working on an algorithmic implementation of a qualitative model for productivity which will continue as an independent study in Spring 2019
\item Project overseen by Prof. Timothy O'Donnell
\end{itemize}

\divider

\cvevent{Research Assistant}{\href{https://opalmedapps.com/}{Opal Med Apps} - a patient portal to empower cancer patients}{May 2018 -- August 2018}{Montreal, Canada}
\begin{itemize}
\item Worked with the Cancer Mission Patient's society to help illustrate their goals and mission statement with visual media
\item Project overseen by Profs. John Kildea and Laurie Hendren
\end{itemize}

\divider

\cvsection{Awards}
\cvachievement{Arts Undergraduate Research Award}{Given to support undergraduate students who undertake research during the summer under the direct supervision of a faculty member}{\textit{\$4,000 grant for work with .txtLab, 2019}}

%\divider

\clearpage

\end{document}
